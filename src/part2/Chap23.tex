\label{ExpRO}\chapter{Expérimentations et résultats observés}

L’étude a comparé les modèles et les
approches selon des critères de précision et de rappel.

\begin{table}
        \centering
        \begin{tabular}{lllll}
            \toprule
            \textbf{Approche} & \textbf{Modèle} & \textbf{Précision} & \textbf{Rappel} & \textbf{Observations} \\
            \midrule

  \textbf{Zero-shot}


   &
  Text-Davinci-003


   &
  ~50%


   &
  78%


   &
  Nombre élevé de faux positifs.


   \\

  \textbf{Few-shot}


   &
  Code-Davinci-002


   &
  ~55%


   &
  70%


   &
  Meilleure contextualisation des
  vulnérabilités.


   \\

  \textbf{Fine-tuning}


   &
  CodeBERT


   &
  59%


   &
  63%


   &
  Meilleur équilibre entre détection et
  précision.


   \\
            \bottomrule
        \end{tabular}
        \caption{}
        \label{tab:}
    \end{table}


Les résultats montrent que \textbf{CodeBERT
fine-tuné} est la solution la plus efficace pour la détection des
vulnérabilités en temps réel.



De plus, lorsqu’il est intégré dans \textbf{VSCode},
l’outil a permis une \textbf{réduction de 80\% des vulnérabilités détectées}
pendant l’édition du code, et jusqu’à \textbf{90\% pour du code généré
automatiquement par GitHub Copilot}.




