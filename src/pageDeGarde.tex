\begin{titlepage}
    \begin{center}
        % Deux logos
%        \includegraphics[width=0.30\textwidth]{figures/placeholder}\hspace{0.5cm}\includegraphics[width=0.30\textwidth]{figures/placeholder}\\[0.9cm]
        % Un logo
        \includegraphics[width=0.45\textwidth]{figures/logos/Logo_INSA_CVL}\\[0.9cm]

        \textsc{\LARGE Institut National des Sciences Appliquées}\\[0.35cm]
        %\textsc{\LARGE Entreprise-Institut 2}\\[1.5cm]

        \HRule \\[0.4cm]

            {\huge \bfseries Transformer-based Vulnerability Detection in Code at EditTime :}\\[0.30cm]
            {\huge \bfseries Zero-shot, Few-shot, or Fine-tuning ?}\\[0.30cm]
            {\Large Étude Bibliographique}\\[0.30cm]

        \HRule \\[1cm]

        % Auteurs et relecteurs
        \begin{minipage}{0.4\textwidth}
            \begin{flushleft} \large
                \emph{Auteurs de l'article :}\\
                    Aaron \textsc{Chan}\\
                    Anant \textsc{Kharkar}\\
                    Roshanak \textsc{Zilouchian Moghaddam}\\
                    Yevhen \textsc{Mohylevskyy}\\
                    Alec \textsc{Helyar}\\
                    Eslam \textsc{Kamal}\\
                    Mohamed \textsc{Elkamhawy}\\
                    Neel \textsc{Sundaresan}\\[0.3cm]
                \emph{Auteurs de l'étude :}\\
                    Mohamed \textsc{Mokrani}\\
                    Lamiaa \textsc{Benejma}\\
                    Mouna \textsc{El Arraf}\\
                    Thomas \textsc{Aubin}\\

            \end{flushleft}
        \end{minipage}
        \begin{minipage}{0.4\textwidth}
            \begin{flushright} \large
                \emph{Responsable du module :} \\
                Pascal \textsc{Berthomé}\\
                \emph{Relecteurs :} \\
                Prénom \textsc{Nom}\\
                Prénom \textsc{Nom}\\
            \end{flushright}
        \end{minipage}
        \vfill
%        \fbox{\includegraphics[width = 0.8\textwidth]{figures/placeholder}}
%        \\[3mm]
        Si la détection de vulnérabilités logicielles est désormais universellement assistée par
        IA et en particulier par l'utilisation de LLM, ces derniers présentent une marge de
        progression importante lors des phases de développement. Nous étudions en quoi les
        approches proposées par l'article sont novatrices \\
        \textbf{Mots-clés :} Transformeurs, Vulnerabilités logicielles, Détection de
    vulnérabilités\\
        \vfill
        {\large \today}
%        \begin{flushright}
%            \small{Modèle \LaTeX    créé par Thomas AUBIN}
%        \end{flushright}
    \end{center}
\end{titlepage}