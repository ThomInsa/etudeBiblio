\chapter{Suggestions d'améliorations et directions futures}



\section{Perspectives et améliorations possibles}



L’article ouvre la voie à de nombreuses améliorations pour l’avenir de la détection de
vulnérabilités en temps réel :



\begin{itemize}

 \item \textbf{Réduction des faux positifs} : L’un des principaux défis des outils
     automatisés est leur tendance à générer des alertes inutiles. La
     combinaison de méthodes basées sur des règles et des modèles
     d’apprentissage profond pourrait améliorer leur précision.

\end{itemize}


\section{Extension aux langages et frameworks variés}



L’étude se concentre principalement sur quelques langages de programmation. Une
généralisation à d’autres langages (Rust, Go, Swift) et frameworks (React,
Angular, Spring) permettrait d’accroître l’utilité de ces outils.

\section{Étude de l’impact en entreprise}

Pour évaluer
réellement l’efficacité de ces modèles, une étude plus approfondie dans des
contextes industriels est nécessaire. Cela permettrait de comprendre :

\begin{itemize}

 \item L’acceptation
     par les équipes de développement.

 \item L’impact
     sur la productivité et la sécurité des applications.

 \item Les
     ajustements nécessaires pour un déploiement à grande échelle.

\end{itemize}



