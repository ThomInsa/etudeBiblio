Les logiciels non sécurisés coûtent très cher aux entreprises, en dépit des moyens très
avancés (outils, recherche) qui s’emploient à la recherche de vulnérabilités. En effet, une
grande partie des vulnérabilités échappe aux outils de détection existants, laissant des
failles potentielles de sécurité pour les utilisateurs et pour les systèmes d’information. Pour
aggraver la situation, les délais entre la création effective de la vulnérabilité dans le code et
sa découverte et correction effectives sont longs. La plupart des solutions d’analyse
existantes exigent pour cela un code compilable et exécutable. Ce retard crée la même
vulnérabilité mais à l’effet de la correction de plus grande complexité.
Les millions de dépôts publics présents sur la plateforme GitHub traduisent cette montée en
flèche des surfaces d’attaques, qui fait de la détection proactive une démarche
incontournable pour déceler les vulnérabilités. Parallèlement, les modèles architecturaux
pour toutes les applications devenant de plus en plus complexes, grâce aux frameworks,
bibliothèques tierces et micro-services qui en multipliant les points d’injection potentielles,
ouvrent de nouveaux vecteurs d’attaques.
Dans un tel contexte, il apparaît clairement que le nouvel objectif est la détection des
vulnérabilités dès l’EditTime du code, même s’il est partiel ni syntaxiquement erroné. Pour
cela, l’objectif est d’offrir une détection interactive immédiate aux développeurs., afin qu’ils
aient la possibilité de corriger les défauts de sécurité survenus dès leur apparition, sans
attendre la phase de compilation/exécution.
Ce concept s’inscrit parfaitement dans l’approche ‘shift-left’ en cybersécurité, inscrite dans le
cœur du DevSecOps, qui vise à intégrer la sécurité le plus tôt possible dans le cycle de
développement. Notons également qu’un bon nombre de vulnérabilités peut être imputé à la
mauvaise connaissance ou la méconnaissance de la sécurité par les développeurs. Ainsi,
en ayant une détection de failles dès le début du processus, il serait possible de réduire les
risques et, par voie de conséquence, d’améliorer la sécurité des logiciels en général, grâce à
une sensibilisation accrue.
    \begin{center}Comment concevoir des outils capables de détecter de manière efficace et en
    temps
    réel les
éventuelles vulnérabilités dans du code incomplet, afin à la fois de réduire les délais de
correction, arrondir la qualité logicielle mais aussi de sécuriser le code dès les premières
phases de son développement ?\end{center}

\chapter{Présentation du domaine et des enjeux en cybersécurité}
    La cybersécurité prend une importance centrale alors que notre société s’automatise, la
première vulnérabilité concernée est le logiciel, qui, comme tout produit technologique,
souffre des défauts incriminés susceptibles de compromettre la sécurité des données, des
utilisateurs et des systèmes. Pour faire face à ce défi majeur, deux techniques du logiciel de
sécurité ont traditionnellement la faveur des experts, il s’agit de l’analyse dynamique ou de
l’analyse statique. L’analyse dynamique est fondée sur l’exécution effective du code,
révélant les vulnérabilités selon le comportement observé, mais souffre d’un fort problème
de couverture persistant, rendant peu réalisables un trop grand nombre d’exécutions
possibles du logiciel. L’analyse statique s’intéresse, elle, au code source ou le code binaire
sans exécuter le logiciel, favorisant ainsi la couverture. Elle nécessite cependant la mise en
place d’une définition manuelle des règles par des experts sur le terrain, ce qui rend
compliqué un accompagnement en tant que stratégie de mesure de la sécurité, lourd
d’efforts et donc ruineux eu égard à l’accélération de l’apparition de nouvelles menaces.
En réponse à ces difficultés, l’intelligence artificielle, et singulièrement le machine learning et
le deep learning, offre des voies prometteuses. En effet, les méthodes traditionnelles de
machine learning reposent sur un ensemble de features extraites manuellement, provoquant
une dépendance à l’humain et des charges lourdes au niveau de la maintenance des
modèles. En revanche, les techniques de deep learning, et tout particulièrement celles
basées sur des modèles de type transformers amènent à une automatisation des
apprentissages des motifs de vulnérabilité directement à partir de grandes bases de
données de code.
Ces nouvelles méthodes constituent de réelles avancées en ce sens qu’elles permettent
d’accroître très nettement la couverture et la précision des détections. Elles sont en outre
bien plus adaptées à l’analyse de code auto-généré par les outils d’assistance au
développement tels que GitHub Copilot ou Codex. Le raisonnement s’appuie sur l’hypothèse
selon laquelle malgré leur indéniable efficacité, ces modèles d’IA générative peuvent
introduire des vulnérabilités dans les logiciels, ce que rend d’autant plus problématique
l’absence de contrôle.
Une autre constante d’évolution est celle de la menace. Auparavant, elle pouvait se
considérer comme technique, actuellement elle ne l’est plus, comme ces vulnérabilités
exploitées massivement par des bots, ou ces attaques utilisant en toute discrétion des failles
de type 0-day pour lesquelles aucun correctif n’existe sous forme de patch. La sécurité
logicielle devient une exigence autant réglementaire, à travers des normes telles l’ISO/IEC
27001, la directive européenne NIS2, ou encore de la FDA pour les logiciels médicaux, qui
avenant la nécessité d’être vigilant et du besoin de solidité des systèmes informatiques.
De surcroît, la diversité et la multitude de vulnérabilités sont sources de complexité : le
nombre de nouvelles vulnérabilités dans la base de données CVE/NVD dépasse le millier
tous les ans, ce qui ne peut se gérer (et réagir) qu’avec des outils, humains, mais pas
uniquement. D’où la nécessité accrue dans la stratégie des entreprises d’intégrer les
modèles de langage (LLMs) en soutien au développement en intégrant la sécurité des
solutions dès les premières étapes de développement, pour protéger efficacement les
utilisateurs finaux et réduire de façon drastique le risque en production.