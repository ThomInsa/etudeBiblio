\chapter{Importance de la détection des vulnérabilités}
Dès lors, énoncé de manière générale, la détection anticipative des vulnérabilités
coutumières à se loger dans les logiciels paraît un souci essentiel, qui participe de la
sécurité des programmes au bénéfice ultime de l’utilisateur final. Un point qui est précisé par
de nombreuses études est que le temps de réaction suivant la vulnérabilité et la qualité de
sa correction sont le plus souvent corrélés aux coûts de son exploitation par un hacker.
D’après un rapport de l’IBM Security « Cost of a Data Breach 2023 » le coût d’une violation
de données serait en moyenne de 4,45 millions de dollars actuellement, une valeur en
constante hausse! Il est vrai que plus une vulnérabilité est anciennement mise au jour, plus
elle est difficile et donc plus coûteuse à combler. Une étude, parue en 2021 dans le Journal
of Systems and Software semble le prouver, puisque dans le cas d’une vulnérabilité repérée
au sein d’un logiciel dans sa phase de mise au point initiale par exemple, son coût peut être
30 fois plus faible en cas d’intervention qu’à l’issue du processus de production phase au
cours de laquelle elle a pu être exposée à l’exploitation en ligne active par les
cybercriminels. Le coût ici correspond aux coûts techniques du remplacement de code
informatique mais aussi aux pertes économiques indirectes à l’exploitation de la
vulnérabilité.
Il existe une multitude de vulnérabilités critiques, telles les injections SQL, les Cross-Site
Scripting (XSS), et les erreurs dans la validation d’un code qui se révèlent dès les premières
écritures du code. Selon une étude menée par OWASP (Open Web Application Security
Project) plus de 70\% des vulnérabilités référencées au sein du Top 10 de OWASP 2021
auraient pu être repérées et corrigées dès la saisie de code (EditTime). La détection en ligne
de leurs vulnérabilités permet ainsi d’en corriger les failles , et in fine de réduire fortement,
les coûts d’intervention ultérieurs, tout en bénéficiant de l’amélioration substantielle de la
qualité du code produit.
En plus d’épargner ces coûts, une telle anticipation devient précieuse pour réduire la fenêtre
d’exposition, aux cyberattaques. Actuellement, d’après le rapport de Veracode “State of
Software Security 2023”
, le temps moyen allant de l’introduction d’une vulnérabilité à sa
prise en compte dépasse généralement les 200 jours. Les applications sont alors
longuement laissées, exposées, aux menaces numériques tiers, malveillantes. En
réagissant à la découverte du défaut au moment même où celui s’introduit, la menace étant
prise en compte dans les heures voire les en temps réel suivant, la protection se renforce
pour les données sensibles, et une continuité des services est préservée.
Les effets d’une exploitation de faille en production sont lourdement dommageables,
exposant la fuite de données sensibles, le choc d’une réputation sévèrement atteinte, des
amendes parfois très éprouvantes comme celles de l’application du RGPD (jusqu’à 4% du
chiffre d’affaires annuel consolidé d’une entreprise) et des pertes d’activité en conséquences
parfois existentielles. Par ailleurs, selon l’étude de Gartner (2022), 60% des petites
entreprises sont déclarées disparues dans les six mois suivant une cyberattaque majeure du
fait des coûts générés pourtant directs en premier lieu comme indirects en second lieu.
En outre, enfin, en adoptant une stratégie de détection des vulnérabilités en EditTime, on
peut avoir l’ambition d’une vraie valeur pédagogique, ayant en interaction directe et
immédiate avec les bonnes pratiques et erreurs potentielles des développeurs et prenant
ainsi une disponibilité sans cesse accrue de la conscience et du niveau leurs compétences
en sécurité applicative. En définitive cela promeut une culture pro-active et résistante de sécurité logicielle au sein des équipes, par là même réduisant cumulativement la dette
technique, suffisant ainsi à consolider la résilience à long terme des systèmes d’information.